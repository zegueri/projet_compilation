\documentclass[a4paper,11pt]{article}
\usepackage[utf8]{inputenc}
\usepackage[T1]{fontenc}
\usepackage{listings}
\usepackage{amsmath}
\usepackage{fullpage}
\title{Mini-interpréteur logique}
\author{Projet compilation}
\date{}
\begin{document}
\maketitle
\section*{Introduction}
Ce mini-projet met en \oe uvre un interpréteur capable de manipuler des fonctions logiques définies soit par une formule, soit par une table de vérité.  Il s'appuie sur \texttt{flex} et \texttt{bison} pour l'analyse lexicale et syntaxique.

\section*{Structures de données}
Les fonctions sont stockées dans une structure~:\newline
\begin{lstlisting}[language=C]
typedef struct {
    char name[MAX_NAME];
    int  arity;                     /* nombre de variables */
    char vars[MAX_VARS][MAX_NAME];  /* noms de variables   */
    int  num_entries;               /* taille de la table  */
    unsigned char table[1<<MAX_VARS];
    char *formula;                  /* texte de la formule */
} Function;
\end{lstlisting}
Un tableau fixe de \verb|Function| est maintenu (\verb|MAX_FUNCS|), chaque entrée décrivant une fonction.  Les tables de vérité sont stockées directement sous forme de tableau de booléens; lorsqu'une fonction est créée à partir d'une formule, celle-ci est évaluée pour remplir la table et sa représentation textuelle est mémorisée.

Les formules booléennes sont d'abord représentées par des arbres dont les nœuds sont de type~:\newline
\begin{lstlisting}[language=C]
enum NType { N_CONST, N_VAR, N_NOT, N_AND, N_OR, N_XOR, N_IMPL, N_FUNCALL };
struct Node {
    enum NType type;
    int val, vidx;
    struct Node *l, *r;
    char *fname;         /* appel de fonction */
    struct Node **args;  /* arguments */
    int argc;
};
\end{lstlisting}
Ces arbres sont produits par l'analyse syntaxique puis évalués pour remplir la table de vérité associée à la définition.

\section*{Fonctionnalités réalisées}
\begin{itemize}
    \item \textbf{define~:} définition d'une fonction via une formule ou une table.
    \item \textbf{list~:} affichage des fonctions connues.
    \item \textbf{varlist~:} liste des variables d'une fonction.
    \item \textbf{table~:} impression de la table de vérité.
    \item \textbf{eval~:} évaluation d'une fonction sur des valeurs données.
    \item \textbf{formula~:} génération d'une formule équivalente à une fonction (depuis la définition ou une forme normale).
    \item Possibilité de lire les commandes depuis un fichier plutôt qu'en mode interactif.
    \item Appel de fonctions déjà définies à l'intérieur d'une nouvelle définition.
\end{itemize}
L'interpréteur reconnaît les mots-clés sans tenir compte de la casse, gère jusqu'à huit variables et peut manipuler une centaine de fonctions simultanément.  Des tests automatiques sont fournis (\verb|tests/|) et validés par \verb|make test|.

\section*{Limites et améliorations possibles}
\begin{itemize}
    \item La gestion des erreurs reste minimale.  Certaines fautes provoquent l'arrêt brutal du programme.
    \item Les tables sont limitées à huit variables pour simplifier les structures et les calculs.
    \item Aucune simplification n'est effectuée lors de la génération automatique d'une formule à partir d'une table; on obtient une forme normale disjonctive potentiellement longue.
    \item Pour étendre le projet, on pourrait améliorer l'analyse des erreurs, ajouter des options pour charger ou sauvegarder les définitions, ou supporter un plus grand nombre de variables.
\end{itemize}

\end{document}
